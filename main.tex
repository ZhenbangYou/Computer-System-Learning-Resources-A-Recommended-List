\documentclass{article}
\usepackage[utf8]{inputenc}
\usepackage{geometry}
\geometry{a4paper,left=3.18cm,right=3.18cm,top=2.54cm,bottom=2.54cm}
\usepackage{hyperref}
\usepackage{CJKutf8}
\usepackage{gbt7714}


\title{\Huge{Computer System Learning Resources}\\ \huge{A Recommended List}}
\author{Zhenbang You}
\date{January 11, 2022}

\begin{document}

\maketitle

\centerline{\large{Last Updated: April 29, 2022}}

\vspace{20pt}

The latest version can always be found at \href{https://www.overleaf.com/read/txqjnjxyxqqx
}{https://www.overleaf.com/read/txqjnjxyxqqx
}

\section{Preliminary}

\textbf{Disclaimer}
\begin{itemize}
    \item This text focuses on \emph{breath} rather than \emph{depth}.
    \item This text mainly focuses on \emph{software systems}, with less concentration on topics such as \emph{VLSI}, \emph{storage systems}, and \emph{embedded systems}, although these are mentioned in some of the following resources.
\end{itemize}

\noindent
\textbf{Directions}
\begin{itemize}
    \item The most fundamental components of computer systems are:
    \begin{enumerate}
        \item Computer Architecture
        \item Operating Systems (OS)
        \item Computer Networks
        \item Compilers
        \item Programming Languages (PL)
    \end{enumerate}
    Master them before moving on to other parts.
    \item The resources in each section are list in an order \emph{suitable for learning}.
    \begin{itemize}
        \item If time is limited, you may just read the first (or the first few) books, and move on to the next section.
        \item To understand some advanced books, knowledge of subsequent sections may be required.
    \end{itemize}
    \item The references help you identify the \emph{authors} rather than \emph{latest versions}.
    \item Given the rapid development of computer science, always read the latest version.
    \begin{itemize}
        \item Tip: search \emph{Amazon} for the latest version. The version number in this text may be outdated with new publications.
    \end{itemize}
\end{itemize}

\noindent
\textbf{How to Read an Abstruse Book?}\\
Three passes.
\begin{enumerate}
    \item Fast: get the main idea and leave out all the details.
    \item Exhaustive: understanding the logic and virtually all the details; few really difficult points can be left out.
    \item Deep: raise questions, and focus on interesting or difficult points.
\end{enumerate}
Here is a famous paper called ``How to Read a Paper" which discusses a similar issue:\\
\href{https://web.stanford.edu/class/cs245/readings/how-to-read-a-paper.pdf}{https://web.stanford.edu/class/cs245/readings/how-to-read-a-paper.pdf}

~

\noindent
\textbf{Why do you recommend so many resources about application?}\\
Learn how to use it before learning how to build it.

~

\noindent
\textbf{Prerequisites}
\begin{itemize}
    \item CSAPP (3rd Edition) \cite{bryant2015computer}
    \item C/C++
    \item Data Structures and Algorithms
    \begin{itemize}
        \item Introduction to Algorithms (4th Edition) \cite{cormen2009introduction}.
        \item In addition to classical algorithms, four kinds of algorithms are widely used in computer systems and thus noteworthy:
        \begin{itemize}
            \item Randomized algorithms
            \item Parallel and distributed algorithms
            \item Approximation algorithms
            \item Online algorithms
        \end{itemize}
    \end{itemize}
    \item Probability
    \begin{itemize}
        \item Elements of \emph{stochastic processes}, especially some basic conclusions and corollaries, are helpful.
    \end{itemize}
    \item Logic, Automata and Complexity
\end{itemize}

\noindent
\textbf{What if I want more?}
\begin{enumerate}
    \item Search for more courses offered by top universities.
    \begin{itemize}
        \item \textbf{How to find them?}
        \begin{enumerate}
            \item For a given university, search for its course list (note that the keyword you are searching for may not appear in the course names).\\
            Example:
            \begin{itemize}
                \item Stanford:
                \href{https://cs.stanford.edu/academicz/courses}{https://cs.stanford.edu/academicz/courses}
                \item UC Berkeley:
                \href{https://www2.eecs.berkeley.edu/Courses/CS/}{https://www2.eecs.berkeley.edu/Courses/CS/}
                \item MIT:
                \href{http://catalog.mit.edu/subjects/6/}{http://catalog.mit.edu/subjects/6/}
            \end{itemize}
            \item To save time by scanning the entire course list, you can find the curriculum or program sheet of the Bachelor's/Master's degree.\\
            Example:
            \begin{itemize}
                \item Stanford Computer Science Master's Program Sheets:\\
                \href{https://cs.stanford.edu/academicz/current-masters/masters-program-sheets/programsheets}{https://cs.stanford.edu/academicz/current-masters/masters-program-sheets/programsheets}
                \item UC Berkeley CS Major Degree Requirements (Undergraduate):
                \begin{enumerate}
                    \item Lower Division:\\
                    \href{https://eecs.berkeley.edu/resources/undergrads/cs/degree-reqs-lowerdiv}{https://eecs.berkeley.edu/resources/undergrads/cs/degree-reqs-lowerdiv}
                    \item Upper Division:\\
                    \href{https://eecs.berkeley.edu/resources/undergrads/cs/degree-reqs-upperdiv}{https://eecs.berkeley.edu/resources/undergrads/cs/degree-reqs-upperdiv}
                \end{enumerate}
            \end{itemize}
            \item The naming convention of course numbers helps you save time.
        \end{enumerate}
        Besides, courses provide \emph{practice} opportunities, which is key to learning computer systems.
    \end{itemize}
    
    \item Find cross-cutting areas like machine learning systems.
    You can always find cutting-edge technologies and research hotspots here.
    Besides, computer systems are extremely powerful when different modules work together.
    \begin{itemize}
        \item Again, searching for advanced courses helps you discover these areas!
    \end{itemize}
\end{enumerate}

\noindent
\textbf{How to Avoid Forgetting?}
\begin{itemize}
    \item Repetition (multiple passes)
    \item Practice
    \item Questioning
    \item Discussion
\end{itemize}

\section{Computer Architecture}
\begin{enumerate}
    \item Computer Organization and Design: The Hardware-Software Interface\\
    (5th Edition \cite{patterson2013computerMIPS}/RISC-V Edition \cite{pattersoncomputerRV}/ARM Edition \cite{patterson2016computerARM})
    \begin{itemize}
        \item The original version is based on MIPS.
        RISC-V version is recommended; after reading this version, you may proceed on ARM version which is a good textbook for learning ARM.
    \end{itemize}
    \item Digital Design and Computer Architecture\\
    (2nd Edition \cite{harris2015digitalMIPS}/RISC-V Edition \cite{harris2021digitalRV}/ARM Edition \cite{harris2015digitalARM})
    \begin{itemize}
        \item You may just read Chap 1-5.
        \item The original version is based on MIPS.
    \end{itemize}
    \item The RISC-V Reader An Open Architecture Atlas \cite{patterson2017risc}
    \item Computer Architecture: A Quantitative Approach (6th Edition) \cite{hennessy2018computer}
    \begin{itemize}
        \item You may leave out appendices the first time you read this book.
        \item Difficult as it may be, this book is just ``the second book for novices".
        If you want to have a deep understanding of a specific topic, do go to read official tutorials/documentations such as those of NVIDIA.
        \item When having some experiences on \emph{parallel computing}, read corresponding chapters of this book again, you will surely gain some new understanding.
        \item Based on my personal experiences, multiple passes are needed to gain thorough comprehension of this masterpiece.
    \end{itemize}
    \item RISC-V Privileged Architecture (slides)\\
    \href{https://riscv.org/wp-content/uploads/2018/05/riscv-privileged-BCN.v7-2.pdf}{https://riscv.org/wp-content/uploads/2018/05/riscv-privileged-BCN.v7-2.pdf}
    \begin{itemize}
        \item A wonderful slide on RISC-V privileged architecture, as well as the core problem ``what is the privileged architecture".
        \item The video of this lecture can be found at\\ \href{https://www.youtube.com/watch?v=fxLXvrLN5jA}{https://www.youtube.com/watch?v=fxLXvrLN5jA}
        \item Most of the books on computer architecture discuss little about \emph{privileged architecture}, resulting in great difficulties understanding the OS kernel.
        Always keep in mind that \emph{ISA} consists of both the unprivileged architecture and the privileged architecture.
    \end{itemize}
    \item RISC-V specifications:
    \href{https://riscv.org/technical/specifications/}{https://riscv.org/technical/specifications/}
    \begin{itemize}
        \item Elaborate specifications as they may be, they are indeed excellent textbooks for both neophytes and specialists!
        \item Appendix A: RVWMO Explanatory Material of ``The RISC-V Instruction Set Manual Volume I: Unprivileged ISA" is a brilliant tutorial for \textbf{Memory Consistency}!
    \end{itemize}
    \item A New Golden Age for Computer Architecture (a Turing Lecture with full text)\\
    \href{https://cacm.acm.org/magazines/2019/2/234352-a-new-golden-age-for-computer-architecture/fulltext}{https://cacm.acm.org/magazines/2019/2/234352-a-new-golden-age-for-computer-architecture/fulltext}
    \begin{itemize}
        \item The famous Turing Lecture by Hennessy and Patterson.
        What wonderful insights of masters!
    \end{itemize}
    \item Prerequisites:
    \begin{itemize}
        \item PL
        \begin{itemize}
            \item Java.
        \end{itemize}
    \end{itemize}
    \item To practice you knowledge of computer architecture, there are basically two ways:
    \begin{itemize}
        \item VLSI,
        \item Parallel computing.
    \end{itemize}
\end{enumerate}

\section{Operating Systems (OS)}
\begin{enumerate}
    \item Books
        \begin{enumerate}
        \item Operating Systems: Three Easy Pieces:
        \href{https://pages.cs.wisc.edu/~remzi/OSTEP/}{https://pages.cs.wisc.edu/~remzi/OSTEP/}
        \begin{itemize}
            \item Due to the research interest of the authors, this book puts much emphasis on File Systems.
            You may leave out some chapters of this part the first time you read it.
            \item This is just an introductory-level textbook, read more after finish this!
        \end{itemize}
        \item Operating Systems Principles \& Practice (2nd Edition)
        \begin{itemize}
            \item Four volumes:
            \begin{itemize}
                \item Volume I: Kernels and Processes \cite{anderson2014operating1}
                \item Volume II: Concurrency \cite{andersonoperating2}
                \item Volume III: Memory Management \cite{andersonoperating3}
                \item Volume IV: Persistent Storage \cite{andersonoperating4}
            \end{itemize}
        \end{itemize}
        \item Operating Systems Concepts (10th Edition) \cite{peterson1985operating}
        \begin{itemize}
            \item Very up-to-date.
            Can be an alternative to the previous one.
            You do not need to read both.
        \end{itemize}
        \item xv6 source and text:
        \href{https://pdos.csail.mit.edu/6.828/2021/xv6.html}{https://pdos.csail.mit.edu/6.828/2021/xv6.html}
        \begin{itemize}
            \item Hands-on experiences with a real OS is indispensable, and \textbf{xv6} is an excellent starting point!
        \end{itemize}    

        \item Supplemental books
        \begin{enumerate}
            \item Linux Kernel Development (4rd Edition) \cite{robert2018linux}
            \item Understanding the Linux Kernel (3rd Edition) \cite{bovet2005understanding}
            \item Linux Device Drivers (3rd Edition) \cite{rubini2001linux}
            \item Understanding Linux network internals (1st Edition) \cite{benvenuti2006understanding}
        \end{enumerate}

    \end{enumerate}
    
    \item Courses
    \begin{itemize}
        \item Lab-based
        \begin{enumerate}
            \item Stanford CS 140E Operating Systems Design and Implementation
            \begin{itemize}
                \item \textbf{Rust} version:
                \href{https://cs140e.sergio.bz/}{https://cs140e.sergio.bz/}
                \item \textbf{C} version:
                \href{https://github.com/dddrrreee/cs140e-22win}{https://github.com/dddrrreee/cs140e-22win}
            \end{itemize}
            Uniqueness:
            \begin{enumerate}
                \item ``This course differs from most OS courses in that it uses \textbf{real hardware} instead of a fake simulator, and almost all of the code will be written by you."
            \end{enumerate}        
            \item Stanford CS 240LX Advanced Systems Laboratory, Accelerated:\\
            \href{https://github.com/dddrrreee/cs140e-22win}{https://github.com/dddrrreee/cs140e-22win}
            
            Uniqueness:
            \begin{enumerate}
                \item ``Our code will run "bare-metal" (without an operating system) on the widely-used ARM-based raspberry pi."
            \end{enumerate}                    
        \end{enumerate}
        \item Paper-based
        \begin{enumerate}
            \item Stanford CS 240 Advanced Topics in Operating Systems:\\
            \href{http://web.stanford.edu/class/cs240/}{http://web.stanford.edu/class/cs240/}        
        \end{enumerate}

    \end{itemize}

    \item Prerequisites:
    \begin{itemize}
        \item Compulsory:
        \begin{enumerate}
        \item Computer Architecture (in particular, privileged architecture)
        \item Data structures and algorithms
        \end{enumerate}
        \item PL
        \begin{itemize}
        \item \textbf{Java}: JVM, GC, Thread, Monitor.
        \item \textbf{Go}: Goroutine, Channel, CSP (Communication Sequential Process), Asynchrony.
        \end{itemize}
    \end{itemize}
    \item \textbf{Suggestions}:
    \begin{itemize}
        \item four passes to learn \emph{OS}
            \begin{enumerate}
                \item How to use: user/programmer's perspective, top-down.
                This pass is assisted by CSAPP and resources about \emph{Linux Programming}.
                \item How to build (build a usable one): builder's perspective, bottom-up.
                \item How to design (build a good one if there is no compatibility issues): both perspectives.
                This pass can only be down with knowledge of all the major parts of computer systems.
                \item Advanced and cross-cutting issues:
                \emph{distributed systems}, \emph{cloud computing},...
            \end{enumerate}
        \item Always think about the interaction and cooperation between
        \begin{itemize}
            \item OS \& hardware,
            \item OS \& computer networks,
            \item OS \& PL,
            \item OS \& DB.
        \end{itemize}
        Also think about whether their boundary can be and should be redefined.
        
        \textbf{DO NOT} overfit \emph{Linux}!
        Not everything of it is reasonable and well-suited to the current need!
    \end{itemize}

\end{enumerate}

\section{Computer Networks}
\begin{enumerate}
    \item Books
    \begin{enumerate}
        \item Computer Networking: A Top Down Approach (8th Edition) \cite{kurosecomputer}
        \item Supplementary (by W. Richard Stevens)
        \begin{itemize}
            \item UNIX Network Programming
            \begin{itemize}
                \item Volume 1, 3rd Edition: The Sockets Networking API \cite{stevens2018unixvolume1}
                \item Volume 2, 2nd Edition: Interprocess Communications \cite{richard1999unixvolume2}
            \end{itemize}
            \item TCP/IP Illustrated
            \begin{itemize}
                \item Volume 1: The Protocols (2nd Edition) \cite{fall2011tcp}
                \item Volume 2: The Implementation \cite{stevens1996tcp}
                \item Volume 3: TCP for Transactions, HTTP, NNTP, and the UNIX Domain Protocols \cite{stevens2000tcp}
            \end{itemize}
        \end{itemize}        
    \end{enumerate}
    
    \item Courses
    \begin{enumerate}
        \item Stanford CS 144 Introduction to Computer Networking:
        \href{https://cs144.github.io/}{https://cs144.github.io/}
        \item Stanford CS 244 Advanced Topics in Networking:\\
        \href{https://2022-cs244.github.io/schedule/}{https://2022-cs244.github.io/schedule/}
        \item Stanford CS 344 Topics in Computer Networks (a.k.a. Build an Internet Router):\\
        \href{https://cs344-stanford.github.io/}{https://cs344-stanford.github.io/}
    \end{enumerate}

    \item Prerequisites:
    \begin{itemize}
        \item Compulsory:
            \begin{enumerate}
            \item Operating Systems
            \end{enumerate}
        \item PL
        \begin{itemize}
            \item \textbf{Java}.
            \item \textbf{Python}: Similar but much simpler socket interface than POSIX.
            \item \textbf{Go}: RPC.
        \end{itemize}
    \end{itemize}
\end{enumerate}

\section{Compilers}
Compilers should be associated with \emph{PL} and \emph{Computer Architecture}.
\begin{enumerate}
    \item Books
    \begin{enumerate}
        \item Compilers: Principles, Techniques and Tools (2nd Edition) \cite{aho2007compilers}
        \begin{itemize}
            \item ``Dragon Book".
            \item Well-known but a little obsolete; still wonderful for new-comers.
        \end{itemize}
        \item Modern Compiler Implementation in C \cite{appel2004modern}/Java (2nd Edition) \cite{appel2003modern}/ML \cite{appel1998modern}
        \begin{itemize}
            \item ``Tiger Book".
        \end{itemize}
        \item Advanced Compiler Design Implementation \cite{muchnick1997advanced}
        \begin{itemize}
            \item ``Whale Book".
        \end{itemize}
        \item Engineering a Compiler (2nd Edition) \cite{cooper2011engineering}    
    \end{enumerate}
    
    \item Courses
    \begin{enumerate}
        \item Stanford CS 143 Compilers:
        \href{https://web.stanford.edu/class/cs143/}{https://web.stanford.edu/class/cs143/}
        \item Stanford CS 243 Program Analysis and Optimization:\\
        \href{https://suif.stanford.edu/~courses/cs243/}{https://suif.stanford.edu/~courses/cs243/}
    \end{enumerate}


    \item Prerequisites:
    \begin{itemize}
        \item Compulsory:
        \begin{enumerate}
        \item TCS (Theoretical Computer Science)
        \begin{enumerate}
            \item Introduction to the Theory of Computation (3rd Edition) \cite{sipser1996introduction}
        \end{enumerate}
        \item Computer Architecture: ILP (Instruction-Level Parallelism), Memory Hierarchy
        \item Data structures and algorithms
    \end{enumerate}
        \item Recommended:
        \begin{enumerate}
            \item Computer Architecture: DLP (Data-Level Parallelism), TLP (Thread-Level Parallelism)
            \item Operating Systems: Thread, Context Switch
        \end{enumerate}
        \item PL
        \begin{itemize}
            \item \textbf{Java}.
        \end{itemize}
    \end{itemize}
\end{enumerate}

\section{Programming Languages (PL)}
Abbreviations:
\begin{itemize}
    \item OOP: Object-Oriented Programming
    \item FP: Functional Programming
\end{itemize}
\textbf{Language List}\\
Unless specified explicitly, always try the latest version.
\begin{itemize}
    \item \textbf{C/C++} (Do learn the latest version of C++, or at least C++17)
    \begin{itemize}
        \item Tutorials
        \begin{enumerate}
            \item The C Programming Language (2nd Edition) \cite{ritchie1988c}
            \begin{itemize}
                \item ``\textbf{K \& R}"
            \end{itemize}
            \item The C++ Programming Language (4th Edition) \cite{stroustrup2013c++}
        \end{enumerate}
        \item Documentations
        \begin{enumerate}
            \item \href{https://cppreference.com/}{https://cppreference.com/}
            \item Boost C++ Libraries:
            \href{https://www.boost.org/}{https://www.boost.org/}
        \begin{itemize}
            \item Sometime \emph{standard C++ libraries} are not enough.
            In this case, \emph{Boost} may provide additional useful functionality.
            Besides, some functionality of \emph{Boost} may be incorporated into \emph{standard C++ libraries} in the future, as the past shows.
        \end{itemize}
        \end{enumerate}

        \item Programming guidelines
        \begin{itemize}
            \item Scott Meyers ``Effective C++" book series
            \begin{itemize}
                \item Effective C++ (3rd Edition) \cite{meyers2005effective}
                \item Effective Modern C++ (1st Edition) \cite{meyers2014effective}
                \item Effective STL (1st Edition) \cite{meyers2001effective}
            \end{itemize}
            \item C++ Core Guidelines:\\ \href{https://isocpp.github.io/CppCoreGuidelines/CppCoreGuidelines}{https://isocpp.github.io/CppCoreGuidelines/CppCoreGuidelines}
        \end{itemize}
        \item Recommended IDE: VSCode on Linux (especially WSL).
        \item Recommended compilers: \emph{latest} GCC and Clang.
        \begin{itemize}
            \item Currently, \textbf{OpenMP} support of \textbf{Clang 13} needs to be installed separately, which is not the case for \textbf{Clang 12}.
            \item \textbf{GCC} can be built from source.
            \item \textbf{Clang} can by downloaded directly from \href{https://releases.llvm.org/download.html}{https://releases.llvm.org/download.html}
        \end{itemize}
        \item Build tool: CMake (also works for CUDA C++).
        \begin{itemize}
            \item CMake Tutorial:
            \href{https://cmake.org/cmake/help/latest/guide/tutorial/}{https://cmake.org/cmake/help/latest/guide/tutorial/}
        \end{itemize}
        \item \textbf{The coding style of C/C++ is much more important than that of other languages}, due to the legacy, the flexibility and the safety issues of this language.
    \end{itemize}
    \item \textbf{Java} (Java 17 is recommended, or at least, choose an LTS version)\\
    Prerequisite languages: C++.\\
    Pay attention to the comparison with C++, as well as JVM and JIT.
    \begin{itemize}
        \item Tutorials
        \begin{enumerate}
            \item Oracle online tutorial:
        \href{https://docs.oracle.com/javase/tutorial/}{https://docs.oracle.com/javase/tutorial/}
            \item Core Java (10th Edition). Two volumes:
            \begin{itemize}
                \item Volume I: Fundamentals \cite{gvero2013core1}
                \item Volume II: Advanced Features \cite{tarimci2014core2}
            \end{itemize}
            \item Java 8 in Action: Lambdas, Streams, and Functional-style Programming (1st Edition) \cite{urma2014java}
        \end{enumerate}
        \item Documentations
        \begin{enumerate}
            \item \href{https://docs.oracle.com/javase/specs/}{https://docs.oracle.com/javase/specs/}
        \end{enumerate}
        \item Programming guidelines
        \begin{itemize}
            \item Effective Java (3rd Edition) \cite{bloch2008effective}
        \end{itemize}

        \item Recommended IDE: IntelliJ Idea
        \item Build tool (applying to all languages on JVM) : \textbf{Maven} or \textbf{Gradle}.
        \begin{itemize}
            \item Maven in 5 Minutes:\\
            \href{https://maven.apache.org/guides/getting-started/maven-in-five-minutes.html}{https://maven.apache.org/guides/getting-started/maven-in-five-minutes.html}
            \item All documentations can be found at:\\
            \href{https://maven.apache.org/}{https://maven.apache.org/}
            \item Try \textbf{Maven} with \textbf{IntelliJ Idea}.
        \end{itemize}
    \end{itemize}

    \item \textbf{Scala}
    (\textbf{Scala 2} is preferred over \textbf{Scala 3} at the moment for the sake of ecosystem)\\
    Prerequisite languages: Java.\\
    Noteworthy features: FP.
    \begin{itemize}
        \item Tutorials
        \begin{enumerate}
            \item Tour of Scala:
            \href{https://docs.scala-lang.org/tour/tour-of-scala.html}{https://docs.scala-lang.org/tour/tour-of-scala.html}
            \item Scala Book:
            \href{https://docs.scala-lang.org/overviews/scala-book/introduction.html}{https://docs.scala-lang.org/overviews/scala-book/introduction.html}
        \end{enumerate}
        \item Documentations
        \begin{itemize}
            \item Guides and Overviews:
            \href{https://docs.scala-lang.org/overviews/index.html}{https://docs.scala-lang.org/overviews/index.html}
            \item All documentations:
            \href{https://docs.scala-lang.org/}{https://docs.scala-lang.org/}
                \begin{itemize}
                    \item \textbf{Scala 3} documentations can also be found here!
                \end{itemize}
        \end{itemize}
        \item Recommended IDE \& build tool: same as \textbf{Java}
        \item Currently, the ecosystem of \textbf{Scala 2} is much better than that of \textbf{Scala 3}.
        The development \textbf{Scala 3} is still at the early stage, but it is also recommended, especially in the long run.
        \end{itemize}
        
    \item \textbf{Kotlin}
    (\textbf{Kotlin/JVM} is recommended, with \textbf{Kotlin/Native} and \textbf{Kotlin/JS} being immature)\\
    Prerequisite languages: Java, Scala.\\
    Noteworthy features: null safety, modest FP.
    \begin{itemize}
        \item Tutorials
        \begin{enumerate}
            \item Get started with Kotlin:
            \href{https://kotlinlang.org/docs/getting-started.html}{https://kotlinlang.org/docs/getting-started.html}
            \item Kotlin Coroutines:\\
            \href{https://github.com/Kotlin/KEEP/blob/master/proposals/coroutines.md}{https://github.com/Kotlin/KEEP/blob/master/proposals/coroutines.md}
            \begin{itemize}
                \item Wonderful article about the design of stackless coroutines!
            \end{itemize}
        \end{enumerate}
        \item Documentations
        \begin{enumerate}
            \item        \href{https://kotlinlang.org/docs/}{https://kotlinlang.org/docs/}
        \end{enumerate}
        \item Recommended IDE \& build tool: same as \textbf{Java}
    \end{itemize}
    
    \item \textbf{Go}\\
    Prerequisite languages: C++, Java.\\
    Noteworthy features: concurrent programming (goroutine + channel + asynchrony, GMP model), modest OOP (compare the OOP of Java and the OOP of Go).
    \begin{itemize}
        \item Turorials
        \begin{enumerate}
            \item A Tour of Go:
            \href{https://go.dev/tour/welcome/1}{https://go.dev/tour/welcome/1}
            \item Effective Go:
            \href{https://go.dev/doc/effective\_go}{https://go.dev/doc/effective\_go}
            \item Official tutorials (including those for generics, fuzzing and accessing a relational database:
            \href{https://go.dev/doc/tutorial/}{https://go.dev/doc/tutorial/}
            \item The Go Blog:
            \href{https://go.dev/blog/}{https://go.dev/blog/}
            \begin{itemize}
                \item Lots of wonderful articles about the design of Go!
            \end{itemize}        
        \end{enumerate}
        \item Documentations
        \begin{enumerate}
            \item All the documentations can be found at:
            \href{https://go.dev/doc/}{https://go.dev/doc/}
        \end{enumerate}
        \item Recommended IDE: GoLand
        \item Build tool:
        \href{https://go.dev/doc/modules/gomod-ref}{https://go.dev/doc/modules/gomod-ref}
    \end{itemize}
    
    \item \textbf{Rust}\\
    Prerequisite languages: C++17, Go, a functional language (e.g., Scala, Haskell, OCaml).\\
    Noteworthy features: ownership, borrow checker, various \textbf{safety}, modest OOP.\\
    Do compare Rust with other languages.
    \begin{itemize}
        \item Tutorials
        \begin{enumerate}
            \item The Rust Programming Language:
            \href{https://doc.rust-lang.org/book/}{https://doc.rust-lang.org/book/}
            \begin{itemize}
                \item Also a good book about the design of programming languages!
            \end{itemize}
            \item Rust by Example:
            \href{https://doc.rust-lang.org/stable/rust-by-example/}{https://doc.rust-lang.org/stable/rust-by-example/}
            \item Asynchronous Programming in Rust:
            \href{https://rust-lang.github.io/async-book/}{https://rust-lang.github.io/async-book/}
            \item The Rustonomicon:
            \href{https://doc.rust-lang.org/nomicon/}{https://doc.rust-lang.org/nomicon/}
            \begin{itemize}
                \item This book is about ``Unsafe Rust".
            \end{itemize}
        \end{enumerate}
        \item Documentations
        \begin{enumerate}
            \item The Rust Reference:
            \href{https://doc.rust-lang.org/reference/}{https://doc.rust-lang.org/reference/}
            \item \href{https://doc.rust-lang.org/beta/}{https://doc.rust-lang.org/beta/}
        \end{enumerate}
        \item Recommended IDE: IntelliJ Idea
        \item Build Tool: Cargo.
        \begin{itemize}
            \item Its tutorial can be found in ``The Rust Programming Language".
        \end{itemize}
    \end{itemize}
    \item \textbf{Haskell}\\
    \emph{Purely functional language}.
    \begin{itemize}
        \item Tutorials
        \begin{enumerate}
            \item Get Started:
            \href{https://www.haskell.org/}{https://www.haskell.org/}
            \item Real World Haskell:
            \href{http://book.realworldhaskell.org/read/}{http://book.realworldhaskell.org/read/}
            \item Learn You a Haskell for Great Good!:
            \href{http://learnyouahaskell.com/chapters}{http://learnyouahaskell.com/chapters}
        \end{enumerate}
        \item All the books, courses, tutorials, documentations and various kinds of resources can be found at:
        \href{https://www.haskell.org/documentation/}{https://www.haskell.org/documentation/}
    \end{itemize}
    \item \textbf{OCaml}
    \begin{itemize}
        \item Tutorials:
        \begin{enumerate}
            \item Real World OCaml:
            \href{https://dev.realworldocaml.org/}{https://dev.realworldocaml.org/}
            \item Official tutorials: \href{https://ocaml.org/learn/tutorials/}{https://ocaml.org/learn/tutorials/}
        \end{enumerate}
        \item Documentations:
        \begin{enumerate}
            \item \href{https://ocaml.org/docs/}{https://ocaml.org/docs/}
        \end{enumerate}        
    \end{itemize}
    \item Prerequisites (for understanding the design of programming languages):
    \begin{itemize}
        \item Compulsory:
        \begin{enumerate}
            \item Computer Architecture
            \item Operating Systems
            \item Compilers
        \end{enumerate}
        \item Recommended:
        \begin{enumerate}
            \item Computer Networks
        \end{enumerate}
    \end{itemize}
    \item \textbf{Courses}
        \begin{itemize}
            \item Stanford CS 242 Programming Languages:
            \begin{itemize}
                \item Autumn:
                \href{https://stanford-cs242.github.io/f19/}{https://stanford-cs242.github.io/f19/}
                \item Winter:
                \href{https://web.stanford.edu/class/cs242/materials.html}{https://web.stanford.edu/class/cs242/materials.html}
            \end{itemize}
            \item Stanford CS 151 Logic Programming:\\
            \href{http://logicprogramming.stanford.edu/stanford/lessons.php}{http://logicprogramming.stanford.edu/stanford/lessons.php}
        \end{itemize}
    \item \textbf{Textbooks}
    \begin{enumerate}
        \item Structure and Interpretation of Computer Programs (2nd Edition \cite{abelson1996structure}/JavaScript Edition \cite{abelson2022structure})
        \begin{itemize}
            \item The original version is based on \textbf{Scheme}, an Lisp dialect.
            However, Lisp and its dialects are obsolete now.
        \end{itemize}
        \item Types and Programming Languages (1st Edition) \cite{pierce2002types}
    \end{enumerate}
    \item \textbf{Suggestions}:
        \begin{itemize}
            \item There is no need to master an entire PL in one shot; instead, study part of it every time you need it.
            \item PLs develop rapidly. To catch up with the latest development, refer to online documentations/tutorials/blogs besides books.
            \item Be sure you have mastered at least one \textbf{modern} language in each of the following paradigms:
            \begin{itemize}
                \item Procedural
                \item Object-Oriented
                \item Functional
            \end{itemize}
            Note that modern languages like \textbf{Go}, \textbf{Scala}, \textbf{Kotlin} and \textbf{Rust} can be greatly different than old ones like \textbf{Java} and \textbf{Python}.
            As a special case, although modern \textbf{C++} (C++11 and later) is really modern, but there are inevitably a great number of legacies, so C++ can be considered as a mixture.
            \item PL can be viewed from at least three perspectives:
            \begin{itemize}
                \item Computer systems
                \item Software engineering
                \item Software theory
                \begin{itemize}
                    \item e.g., relationship between programming paradigms and Church-Turing thesis.
                \end{itemize}
            \end{itemize}
            Therefore, you can always find lots of cross-cutting issues in PL.
            \item A complete list of \textbf{programming paradigms}:\\
            \href{https://en.wikipedia.org/wiki/Programming\_paradigm}{https://en.wikipedia.org/wiki/Programming\_paradigm}
            \item Pay attention to three kinds of \emph{safety}
            \begin{itemize}
                \item Type safety
                \item Memory safety
                \item Thread safety
            \end{itemize}
            \item Pay attention to the \emph{memory consistency model} (always called \emph{memory model} in this context) of each language, although this virtually has no impact on \emph{application-level programming}.
            \item \emph{Concrete examples} always help a lot for understanding \emph{abstract concepts}, and this is also one of the reasons why you should master several languages.
            \item Pay attention to the interoperation between a certain language with C/C++ (and languages on JVM with Java).
            \item Only IDEs I have used will be listed here, so there may be other wonderful IDEs.
        \end{itemize}
\end{itemize}

\section{Parallel Computing}
\emph{Parallel computing} consists of three components:
\begin{itemize}
    \item Architecture (system)
    \item Programming (application)
    \item Algorithms (theory)
\end{itemize}
\emph{Parallel computing} is an natural extension to \emph{computer architecture}.\\
\textbf{Special Notes}
\begin{itemize}
    \item Do read \emph{Computer Architecture: A Quantitative Approach (CAAQA)} before diving into this, and revisit that masterpiece after having some hands-on experience of parallel computing!
    \item \emph{Parallel Computing} without \emph{Memory Optimization} is \textbf{ridiculous}!
    \item Do learn how to analyze performance bottleneck, which is key to the success of parallel programs.
    Knowledge like that of \emph{computer architecture} and \emph{operating systems} can be extremely helpful.
    \item \emph{Parallel computing} should include \emph{distributed computing}; however, the latter is not involved in this section.
\end{itemize}
\textbf{Platforms}
\begin{itemize}
    \item \textbf{CUDA} (Category: DLP, GPU)
    \begin{itemize}
        \item CUDA by Example (1st Edition) \cite{sanders2010cuda}
        \begin{itemize}
            \item A little obsolete, but the ideas are still well-presented.
            If your foundation is good enough, go to the following two documentations directly, and these two are highly recommended.
        \end{itemize}
        \item CUDA C++ Programming Guide:\\
        \href{https://docs.nvidia.com/cuda/cuda-c-programming-guide/}{https://docs.nvidia.com/cuda/cuda-c-programming-guide/}
        \item CUDA C++ Best Practices Guide:\\
        \href{https://docs.nvidia.com/cuda/cuda-c-best-practices-guide/}{https://docs.nvidia.com/cuda/cuda-c-best-practices-guide/}
        \item All documentations can be found at:
        \href{https://docs.nvidia.com/cuda/}{https://docs.nvidia.com/cuda/}
    \end{itemize}
    CUDA is also fantastic for \emph{Asynchronous Programming} and \emph{Heterogeneous Programming}!
    You can also have a taste of \emph{Compute Hierarchy} with CUDA!
    \item \textbf{CPU intrinsics} (Category: DLP, CPU)
    \begin{itemize}
        \item \textbf{x86 intrinsics} (Category: DLP, multimedia SIMD instruction set extensions)\\
        \href{https://www.intel.com/content/www/us/en/docs/intrinsics-guide/index.html}{https://www.intel.com/content/www/us/en/docs/intrinsics-guide/index.html}
        \item \textbf{ARM SVE2 intrinsics} (Category: DLP, vector architecture)\\
        \href{https://developer.arm.com/documentation/102340/0001/Program-with-SVE2}{https://developer.arm.com/documentation/102340/0001/Program-with-SVE2}
        \item \textbf{ARM Neon intrinsics} (Category: DLP, multimedia SIMD instruction set extensions)\\
        \href{https://developer.arm.com/documentation/102467/0100/Why-Neon-Intrinsics-}{https://developer.arm.com/documentation/102467/0100/Why-Neon-Intrinsics-}
    \end{itemize}
    \item \textbf{MPI} (Category: TLP, message passing)
    \begin{itemize}
        \item MPI Tutorial:
        \href{https://mpitutorial.com/tutorials/}{https://mpitutorial.com/tutorials/}
        \item YouTube video
        \begin{itemize}
            \item MPI Basics:
            \href{https://www.youtube.com/watch?v=c0C9mQaxsD4}{https://www.youtube.com/watch?v=c0C9mQaxsD4}
            \item MPI Advanced:
            \href{https://www.youtube.com/watch?v=q9OfXis50Rg}{https://www.youtube.com/watch?v=q9OfXis50Rg}
        \end{itemize}
    \end{itemize}
    \item \textbf{OpenMP} (Catrgory: TLP, shared memory)
    \begin{itemize}
        \item Tim Mattson’s (Intel) “Introduction to OpenMP” (2013) on YouTube
        \begin{itemize}
            \item Video:\\
            \href{https://www.youtube.com/playlist?list=PLLX-Q6B8xqZ8n8bwjGdzBJ25X2utwnoEG}{https://www.youtube.com/playlist?list=PLLX-Q6B8xqZ8n8bwjGdzBJ25X2utwnoEG}
            \item Slides:\\
            \href{https://www.openmp.org/wp-content/uploads/Intro\_To\_OpenMP\_Mattson.pdf}{https://www.openmp.org/wp-content/uploads/Intro\_To\_OpenMP\_Mattson.pdf}
        \end{itemize}
    \end{itemize}
    For those familiar with \emph{pthread}, it is quite easy to learn \textbf{OpenMP}.
\end{itemize}
\textbf{Categories}\\
The essence of \emph{parallel computing} is \emph{the lack of dependencies}.
\begin{itemize}
    \item ILP (Instruction-Level Parallelism)\\
    Mathematical model of ILP -- DAG (Directed Acyclic Graph):
    \begin{itemize}
        \item Node: a stage of a instruction.
        \item Edge: dependency (data dependency, control dependency, name dependency) between a pair of nodes.
    \end{itemize}
    Goal:
    eliminate dependencies (edges), and exploit the lack of dependencies between nodes.
    \item DLP (Data-Level Parallelism)\\
        The essence of DLP programming: \emph{Vectorization}.
        \begin{itemize}
            \item To get some hands-on experiences with this, you may start with \emph{PyTorch}, since this relieves you from some low-level details like remaining elements and memory hierarchy.
            \begin{itemize}
                \item Official tutorial: \href{https://pytorch.org/tutorials/}{https://pytorch.org/tutorials/}
                \item Also, \textbf{PyTorch} is a convenient tool to exploit GPU for parallel computing.
            \end{itemize}
        \end{itemize}
        
    \item TLP (Thread-Level Parallelism)\\
    The essence of TLP programming: \textbf{async} (concurrent control flow), \textbf{await} (synchronization).
    \begin{itemize}
        \item Concurrency mechanisms
        \begin{enumerate}
            \item Thread: C++, Java, pthread
            \item Future/Promise: Scala
            \item Stackful Coroutine: Go
            \item Stackless Coroutine: Kotlin, Rust (async/await)
        \end{enumerate}
        \textbf{Aside}:
        \begin{enumerate}
            \item In essence, every kind of \emph{Concurrency} is an encapsulation of \emph{Asynchrony}.
        Therefore, \emph{Concurrent Programming} can be seen as a subset of \emph{Asynchronous Programming}.
            \begin{itemize}
                \item Here is a good summary of \emph{Asynchronous programming techniques} provided in the tutorial of \emph{Kotlin}:
            \href{https://kotlinlang.org/docs/async-programming.html}{https://kotlinlang.org/docs/async-programming.html}
            \end{itemize}
            \item The essence of \emph{Asynchronous Programming} is \emph{async/await}, as well as \emph{suspension points}.
            \begin{itemize}
                \item For \emph{Threads} and \emph{Stackful Coroutines}, every point is a suspension point, while for \emph{Stackless Coroutines}, only a fraction of points can be suspension points and they are declared explicitly.
            \end{itemize}
            \item For implementations, figure out what is ``\textbf{continuation}" and how it varies in \emph{Processes}, \emph{Threads}, \emph{Stackful Coroutines}, and \emph{Stackless Coroutines}.
            Also think about the relation between \emph{continuation} and \emph{suspension points}.
        \end{enumerate}

        \item Synchronization mechanisms
        \begin{itemize}
            \item Shared memory
            \begin{enumerate}
                \item Mutex, Condition Variable: most languages.
                \item Monitor: Java.
                For C++, this can be readily emulated by \emph{RAII}.
                \item Atomic variables/operations: most languages.
                \item Barrier: most languages, as well as CUDA.
                \item Read Write Lock: most languages.
                This is especially useful in DLP.
            \end{enumerate}
            Many widely used mechanisms are not listed here.
            \item Message passing
            \begin{enumerate}
                \item Channel + Select: Go, Kotlin.
                \item Actor Model:
                    \begin{itemize}
                        \item Akka (with Java/Scala interface):\\ \href{https://doc.akka.io/docs/akka/current/typed/guide/introduction.html}{https://doc.akka.io/docs/akka/current/typed/guide/introduction.html}
                        \item Kotlin:\\
                        \href{https://kotlinlang.org/docs/shared-mutable-state-and-concurrency.html#actors}{https://kotlinlang.org/docs/shared-mutable-state-and-concurrency.html\#actors}
                    \end{itemize}
            \end{enumerate}
            \item High-level encapsulations
            \begin{enumerate}
                \item Thread-safe collections
                    \begin{itemize}
                    \item Java: java.util.concurrent:\\
                    \href{https://docs.oracle.com/en/java/javase/17/docs/api/java.base/java/util/concurrent/package-summary.html}{https://docs.oracle.com/en/java/javase/17/docs/api/java.base/java/util\\/concurrent/package-summary.html}
                    \end{itemize}
            \end{enumerate}
        \end{itemize}
    \end{itemize}
\end{itemize}
\textbf{Courses}
\begin{itemize}
    \item Stanford CS 149 Parallel Computing:
    \href{https://gfxcourses.stanford.edu/cs149/fall21/}{https://gfxcourses.stanford.edu/cs149/fall21/}
    \item Stanford CME 323 Distributed Algorithms and Optimization:\\
    \href{https://stanford.edu/~rezab/classes/cme323/S20/}{https://stanford.edu/~rezab/classes/cme323/S20/}
\end{itemize}
\textbf{Prerequisites}
    \begin{itemize}
        \item Compulsory:
        \begin{enumerate}
        \item Computer Architecture
        \item Compilers
    \end{enumerate}
        \item Recommended:
        \begin{enumerate}
            \item Operating Systems
        \end{enumerate}
    \end{itemize}

\section{Database (DB)}
\textbf{Special Notes}
\begin{itemize}
    \item Think about the relationship (cooperation and conflicts) between DB and OS.
    Also, when facing the same issues (there are lots of them) like concurrency control, how do they handle them respectively?
    \item \emph{Parallel and distributed databases} and \emph{NoSQL} are the hotspots nowadays.
    Do place emphasis on them!
    \item View DB from a full-stack perspective, that is, from systems, to applications, to theory (relation algebra).
    Lots of knowledge you have learned elsewhere plays an important role here.
    Also, you can find plenty of cross-cutting issues here.
    \item Pay attention to the relationship between DB and big data/data mining.
\end{itemize}

\begin{enumerate}
    \item Database System Concepts (7th Edition) \cite{silberschatz2002database}
    \begin{itemize}
        \item Partition:
        \begin{itemize}
            \item Application: Chapter 2-11, 25, 26, 28-32
            \item Systems: Chapter 12-24, 32
            \item Theory: Chapter 2, 27
        \end{itemize}
        As a beginner, you may read chapter 1-7, 12-19 first.        
        \item For Chapter 20-23, basic knowledge of \emph{distributed systems} can be quite helpful.
        See ``Prerequisites" for more advice.
        \item Reflect on Chapter 10-11 after gaining some experiences on big data/cloud computing (e.g., programming with Spark).
        \item Although Chapter 3-5 does teach a lot of SQL and these are really helpful, you should refer to the following manuals if you want to learing more specifications, especially those specific to a certain implementation of SQL.
        \item CMU 15-445 and Stanford CS 145, 245 are almost covered by this book.
        \item Recommended online chapters
        \begin{enumerate}
            \item Chapter 27: Formal-Relational Query Languages.
            Coverage:
            \begin{enumerate}
                \item The tuple relational calculus
                \item The domain relational calculus
                \item \emph{Datalog} (a nonprocedural query language based on the \emph{logic-programming} language \emph{Prolog})
            \end{enumerate}
            \item Chapter 32: PostgreSQL.
            A good case study.
        \end{enumerate}
    \end{itemize}
    \item Courses
    \begin{enumerate}
        \item DB applications
        \begin{itemize}
            \item Stanford CS 145 Data Management and Data Systems:\\
            \href{https://cs145-fa20.github.io/}{https://cs145-fa20.github.io/}
        \end{itemize}
        \item DB systems
        \begin{itemize}
            \item CMU 15-445/645 Intro to Database Systems:\\
            \href{https://15445.courses.cs.cmu.edu/fall2019/schedule.html}{https://15445.courses.cs.cmu.edu/fall2019/schedule.html}
            \item Stanford CS 245 Principles of Data-Intensive Systems:\\
            \href{https://web.stanford.edu/class/cs245/}{https://web.stanford.edu/class/cs245/}
        \end{itemize}
            \item Advanced DB systems
            \begin{itemize}
                \item CMU 15-721 Advanced Database Systems:\\
                \href{https://15721.courses.cs.cmu.edu/spring2020/schedule.html}{https://15721.courses.cs.cmu.edu/spring2020/schedule.html}        
            \end{itemize}
    \end{enumerate}

    \item Query languages and platforms
    \begin{itemize}
        \item SQL
        \begin{itemize}
            \item PostgreSQL
            \begin{enumerate}
                \item Database System Concepts chapter 32:\\
                \href{https://db-book.com/online-chapters-dir/32.pdf}{https://db-book.com/online-chapters-dir/32.pdf}
                \item Tutorial:
                \href{https://www.postgresqltutorial.com/}{https://www.postgresqltutorial.com/}
                \item Documentation \& Manuals:
                \href{https://www.postgresql.org/docs/}{https://www.postgresql.org/docs/}
            \end{enumerate}     
            \item MySQL
            \begin{enumerate}
                \item Tutorial:
                \href{https://www.mysqltutorial.org/}{https://www.mysqltutorial.org/}
                \item Reference manual:
                \href{https://dev.mysql.com/doc/refman/8.0/en/}{https://dev.mysql.com/doc/refman/8.0/en/}
            \end{enumerate}
            \item Which one should I choose?
            \begin{enumerate}
                \item They are two most popular open source databases.
                \item Generally speaking, MySQL is still more popular for historic reasons, while PostgreSQL is more advanced and far more elegant.
            \end{enumerate}
            \item Tips of using specific databases:\\
            \href{https://db-book.com/university-lab-dir/db-tips.html}{https://db-book.com/university-lab-dir/db-tips.html}
        \end{itemize}
        \item NoSQL
        \begin{itemize}
            \item MongoDB
            \begin{enumerate}
                \item Get Started:
                \href{https://www.mongodb.com/docs/guides/}{https://www.mongodb.com/docs/guides/}
                \item Documentations:
                \href{https://www.mongodb.com/docs/}{https://www.mongodb.com/docs/}
            \end{enumerate}
        \end{itemize}
        \item Spark
        \begin{itemize}
            \item Quick Start:
            \href{https://spark.apache.org/docs/latest/quick-start.html}{https://spark.apache.org/docs/latest/quick-start.html}
        \end{itemize}
        \item DB-Engines Ranking:
            \href{https://db-engines.com/en/ranking}{https://db-engines.com/en/ranking}
            \begin{itemize}
                \item Popular database engines and lots of relevant information can be found here.
            \end{itemize}        
        \item Advice:
        \begin{enumerate}
            \item When studying query languages, reason about \emph{declarative programming} (e.g., \emph{SQL}) and \emph{logic programming} (e.g., \emph{Datalog}).
            (Aside: \emph{logic programming} is a subset of \emph{declarative programming}.)
        \end{enumerate}
    \end{itemize}

    \item Prerequisites:
    \begin{itemize}
        \item Compulsory:
        \begin{enumerate}
            \item Data structures and algorithms: those operating data on secondary storage are particularly important
            \item Operating systems
        \end{enumerate}
        \item Recommended:
        \begin{enumerate}
            \item Computer architecture
            \begin{itemize}
                \item Lots of similar ideas in \emph{databases} can be found in \emph{computer architecture})
            \end{itemize}
            \item Computer networks
            \begin{itemize}
                \item required for learning \emph{parallel and distributed databases})
            \end{itemize}
            \item Programming languages
            \begin{itemize}
                \item Query languages help deepening our understanding of differences among programming paradigms (e.g., \emph{imperative} vs \emph{declarative}), as well as some feasible but rarely used programming paradigms (e.g., \emph{logic programming}).
            \end{itemize}
        \end{enumerate}
        \item A few words about \emph{distributed systems}
        \begin{itemize}
            \item It is quite hard to determine which one of the following should be learned first: \emph{parallel and distributed databases} and \emph{distributed systems}.
            My personal feeling is, study of these two fields can be interleaved, since you often need knowledge of the other field when studying either one field.
            As a always reasonable method, study them multiple times so that you will not miss important prerequisite knowledge, if the optimal order of learning is hard to figure out.
        \end{itemize}
        \item PL
        \begin{itemize}
            \item Java: JDBC.
        \end{itemize}
    \end{itemize}
\end{enumerate}

\section{Distributed Computing/Distributed Systems}
\begin{enumerate}
    \item MIT 6.824 Distributed Systems:
    \href{https://pdos.csail.mit.edu/6.824/schedule.html}{https://pdos.csail.mit.edu/6.824/schedule.html}
    \item Supplementary
    \begin{enumerate}
        \item Princeton COS 418 Distributed Systems:\\
        \href{https://www.cs.princeton.edu/courses/archive/fall19/cos418/}{https://www.cs.princeton.edu/courses/archive/fall19/cos418/}
        \item Stanford CS 244B Distributed Systems:
        \href{https://www.scs.stanford.edu/22sp-cs244b/}{https://www.scs.stanford.edu/22sp-cs244b/}
        \item Stanford CS 251 Cryptocurrencies and Blockchain Technologies:\\
        \href{https://cs251.stanford.edu/}{https://cs251.stanford.edu/}
    \end{enumerate}
    
    \item Prerequisites:
    \begin{itemize}
        \item Compulsory:
        \begin{enumerate}
        \item Operating Systems
        \item Computer Networks
        \item Database        
        \end{enumerate}
        \item Recommended:
        \begin{enumerate}
        \item Computer Architecture
        \item Programming Languages
        \end{enumerate} 
    \end{itemize}
\end{enumerate}

\section{Cloud Computing}
\begin{enumerate}
    \item CMU 15-719 Advanced Cloud Computing:\\
    \href{https://www.cs.cmu.edu/~15719/old/spring2019/syllabus.html}{https://www.cs.cmu.edu/~15719/old/spring2019/syllabus.html}
    \item Prerequisites:
    \begin{itemize}
        \item Compulsory:
        \begin{enumerate}
            \item Computer Architecture
            \item Operating Systems
            \item Computer Networks
            \item Distributed Systems
            \item Database
        \end{enumerate}
        \item Recommended:
        \begin{enumerate}
            \item Parallel Computing
        \end{enumerate}
    \end{itemize}
    \item \textbf{Suggestions}
    \begin{enumerate}
        \item There are plenty of \emph{cross-cutting issues}.
        Make sure the prerequisites are met before moving on.
        \item When reading cutting-edge papers concerning \emph{OS}, \emph{computer network} or \emph{computer architecture}, you may encounter issues of \emph{cloud computing}.
        At that moment, you had better have some basic knowledge about it before reading those papers.
    \end{enumerate}
\end{enumerate}

\section{Big Data}
\begin{enumerate}
    \item Designing Data-Intensive Applications: The Big Ideas Behind Reliable, Scalable, and Maintainable Systems (1st Edition) \cite{kleppmann2017designing}
    \item Prerequisites:
    \begin{itemize}
        \item Compulsory
        \begin{enumerate}
            \item Database
            \item Distributed systems
        \end{enumerate}
    \end{itemize}
\end{enumerate}

\section{Software Engineering}
\begin{itemize}
    \item The Mythical Man-Month: Essays on Software Engineering (Anniversary Edition) \cite{brooks1995mythical}
    \item A Philosophy of Software Design \cite{ousterhout2018philosophy}
    \item Materials discussing \emph{best practice} are also valuable, which can be found at the \textbf{PL} section, as well as the \textbf{CUDA} subsection.
\end{itemize}

\section{Program Analysis}
\begin{enumerate}
    \item Software Foundations:
        \href{https://softwarefoundations.cis.upenn.edu/}{https://softwarefoundations.cis.upenn.edu/}
        \begin{itemize}
            \item It consists of 6 volumes.
            \item Volume 1 also teaches \textbf{Coq}.
        \end{itemize}
    \item Prerequisites:
    \begin{itemize}
        \item Compulsory:
        \begin{enumerate}
            \item Compilers
            \item Programming languages
        \end{enumerate}
    \end{itemize}
\end{enumerate}

\section{Computer Security}
Security issues are everywhere.
New security issues arise with new computer technologies invented.
\begin{enumerate}
    \item Stanford CS 155 Computer and Network Security:
    \href{https://cs155.stanford.edu/}{https://cs155.stanford.edu/}
    \item Stanford CS 253 Web Security:
    \href{https://web.stanford.edu/class/cs253/}{https://web.stanford.edu/class/cs253/}
    \item Stanford CS 255 Introduction to Cryptography:
    \href{https://crypto.stanford.edu/~dabo/cs255/}{https://crypto.stanford.edu/~dabo/cs255/}
    \item Stanford CS 355 Advanced Topics in Cryptography:\\
    \href{https://crypto.stanford.edu/cs355/22sp/}{https://crypto.stanford.edu/cs355/22sp/}
    \item Stanford CS 356 Topics in Computer and Network Security:
    \href{https://cs356.stanford.edu/}{https://cs356.stanford.edu/}
    \item Prerequisites:
    \begin{itemize}
        \item Compulsory:
        \begin{enumerate}
            \item Computer architecture
            \item Operating systems
            \item Computer networks
            \item Programming languages
        \end{enumerate}
        Indeed, you can learn a lot about \emph{computer security} in all of these fields.
        As a suggestion, do not leave out the \emph{security} part when learning them.
    \end{itemize}
\end{enumerate}

\section{Instruction Set Architecture (ISA)}
All of the following ISA can be learned by CSAPP or ``Computer Organization and Design: The Hardware/Software Interface".
However, the following resources help you go further.
\begin{itemize}
    \item \textbf{x86}
    \begin{itemize}
        \item Intel® 64 and IA-32 Architectures Software Developer Manuals:\\
        \href{https://www.intel.com/content/www/us/en/developer/articles/technical/intel-sdm.html}{https://www.intel.com/content/www/us/en/developer/articles/technical/intel-sdm.html}
        \item AMD Developer Guides, Manuals \& ISA Documents:\\
        \href{https://developer.amd.com/resources/developer-guides-manuals/}{https://developer.amd.com/resources/developer-guides-manuals/}
    \end{itemize}
    Master both the ``Intel" format and the ``AT\&T" format.
    \item \textbf{RISC-V}
    \begin{itemize}
        \item RISC-V Specifications:
        \href{https://riscv.org/technical/specifications/}{https://riscv.org/technical/specifications/}
        \begin{itemize}
            \item Wonderful textbooks on computer architecture!
        \end{itemize}
    \end{itemize}
    \item \textbf{ARM}
    \begin{itemize}
        \item ARM® CPU Architecture Key Documents:\\
        \href{https://developer.arm.com/architectures/cpu-architecture}{https://developer.arm.com/architectures/cpu-architecture}
        \item Arm® Architecture Reference Manual Supplement Armv9, for Armv9-A architecture profile:
        \href{https://developer.arm.com/documentation/ddi0608/latest}{https://developer.arm.com/documentation/ddi0608/latest}
    \end{itemize}
    \item \textbf{PTX}
    \begin{itemize}
        \item PTX ISA :: CUDA Toolkit Documentation:\\
        \href{https://docs.nvidia.com/cuda/parallel-thread-execution/}{https://docs.nvidia.com/cuda/parallel-thread-execution/}
    \end{itemize}
    \item \textbf{Java Bytecode}
    \begin{itemize}
        \item Java Language and Virtual Machine Specifications:\\
    \href{https://docs.oracle.com/javase/specs/}{https://docs.oracle.com/javase/specs/}
    \end{itemize}
\end{itemize}

\section{Linux Programming}
\begin{enumerate}
    \item Linux man pages
    \begin{itemize}
        \item ``man" command in Linux shell, like ``man fork" or ``man 2 fork" where ``2" specifies the volume.
        \item Linux man pages online:
        \href{https://man7.org/linux/man-pages/}{https://man7.org/linux/man-pages/}
    \end{itemize}
    \item Advanced programming in the UNIX environment (3rd Edition) \cite{stevens1992advanced}
\end{enumerate}

\section{Linking and Loading}
\begin{enumerate}
    \item 
    \begin{CJK}{UTF8}{gbsn}
    程序员的自我修养: 链接, 装载与库 \cite{selfcultivation}
    \end{CJK}
    \item Linkers and Loaders (1st Edition) \cite{levine264linkers}
    \item \textbf{Suggestions}
    \begin{enumerate}
        \item Do not just \emph{read books} or \emph{play with GCC/Clang}.
        Do both together.
        \item The first book plays a primary role, while the second is supplementary (however, I still recommend you read it).
        \item These two books are relatively \textbf{obsolete} (sorry I did not find any relevant modern book).
        Besides, there are lots of academic errors in the first book.
        \item These two books cover lots of knowledge.
        Pick the chapters you need.
        There is no need to read every chapter.
    \end{enumerate}
\end{enumerate}

\section{Tools}
\begin{enumerate}
    \item Git
    \begin{itemize}
        \item Git Magic:
        \href{http://www-cs-students.stanford.edu/~blynn/gitmagic/book.pdf}{http://www-cs-students.stanford.edu/~blynn/gitmagic/book.pdf}
        \item Official documentations:
        \href{https://git-scm.com/doc}{https://git-scm.com/doc}
    \end{itemize}
    \item UNIX Makefile
    \begin{itemize}
        \item Makefile Tutorial By Example:
        \href{https://makefiletutorial.com/}{https://makefiletutorial.com/}
    \end{itemize}
    \item Docker
    \begin{itemize}
        \item Official tutorials and documentations:
        \href{https://docs.docker.com/}{https://docs.docker.com/}
    \end{itemize}
    \item Shell Script
    \begin{itemize}
        \item Shell Scripting Tutorial:
        \href{https://www.shellscript.sh/}{https://www.shellscript.sh/}
        \item The following two tutorials are basically the same and are written by the same author, with the former being more newbie-friendly:
        \begin{itemize}
            \item Bash Scripting Tutorial for Beginners:\\ \href{https://linuxconfig.org/bash-scripting-tutorial-for-beginners}{https://linuxconfig.org/bash-scripting-tutorial-for-beginners}
            \item Bash Scripting Tutorial: \href{https://linuxconfig.org/bash-scripting-tutorial}{https://linuxconfig.org/bash-scripting-tutorial}
        \end{itemize}
        \item GNU Bash manual: \href{https://www.gnu.org/software/bash/manual/}{https://www.gnu.org/software/bash/manual/}
    \end{itemize}
\end{enumerate}

\section{Coding Style}
\begin{itemize}
    \item Google Style Guides:
    \href{https://google.github.io/styleguide/}{https://google.github.io/styleguide/}
    \item Clang-Format Style Options:
    \href{https://clang.llvm.org/docs/ClangFormatStyleOptions.html}{https://clang.llvm.org/docs/ClangFormatStyleOptions.html}
    \begin{itemize}
        \item \emph{VS Code} can automatically format your code with a ``.clang-format" file without installing anything!
    \end{itemize}
\end{itemize}

\section{Miscellaneous}
\begin{enumerate}
    \item Intel® Product Specifications:
    \href{https://ark.intel.com}{https://ark.intel.com}
    \item GCC online documentation:
    \href{https://gcc.gnu.org/onlinedocs/}{https://gcc.gnu.org/onlinedocs/}
    \begin{itemize}
        \item Optimize Options:
        \href{https://gcc.gnu.org/onlinedocs/gcc/Optimize-Options.html}{https://gcc.gnu.org/onlinedocs/gcc/Optimize-Options.html}
        \item Option Summary:
        \href{https://gcc.gnu.org/onlinedocs/gcc/Option-Summary.html}{https://gcc.gnu.org/onlinedocs/gcc/Option-Summary.html}
        \item Instrumentation Summary (including sanitizers and profilers):\\
        \href{https://gcc.gnu.org/onlinedocs/gcc/Instrumentation-Options.html}{https://gcc.gnu.org/onlinedocs/gcc/Instrumentation-Options.html}
    \end{itemize}
    \item GDB documentation:
    \href{https://www.sourceware.org/gdb/documentation/}{https://www.sourceware.org/gdb/documentation/}
    \item GNU Manuals Online:
    \href{https://www.gnu.org/manual/}{https://www.gnu.org/manual/}
    \item Clang 12 documentation:
    \href{https://releases.llvm.org/12.0.0/tools/clang/docs/index.html}{https://releases.llvm.org/12.0.0/tools/clang/docs/index.html}
    \begin{itemize}
        \item Change the version number in the URL to get the documentations of other versions.
    \end{itemize}
    \item LLVM documentation:
    \href{https://llvm.org/}{https://llvm.org/}
    \begin{itemize}
        \item Lots of remarkable projects can be found there!
    \end{itemize}
\end{enumerate}

\bibliographystyle{abbrv}
\bibliography{reference}

\end{document}
